
% Created by Arvind Kumar Mishra
% All the content is owned by Arvind Mishra

\documentclass[11pt,a4paper,sans]{moderncv} % Font sizes: 10, 11, or 12; paper sizes: a4paper, letterpaper, a5paper, legalpaper, executivepaper or landscape; font families: sans or roman

\moderncvstyle{classic} % CV theme - options include: 'casual' (default), 'classic', 'oldstyle' and 'banking'
\moderncvcolor{blue} % CV color - options include: 'blue' (default), 'orange', 'green', 'red', 'purple', 'grey' and 'black'

\usepackage{lipsum} % Used for inserting dummy 'Lorem ipsum' text into the template

\usepackage[scale=0.75]{geometry} % Reduce document margins
\usepackage{marginnote}

%\setlength{\hintscolumnwidth}{3cm} % Uncomment to change the width of the dates column
%\setlength{\makecvtitlenamewidth}{10cm} % For the 'classic' style, uncomment to adjust the width of the space allocated to your name

%----------------------------------------------------------------------------------------
%	NAME AND CONTACT INFORMATION SECTION
%----------------------------------------------------------------------------------------

\firstname{Arvind } % Your first name
\familyname{Mishra} % Your last name

% All information in this block is optional, comment out any lines you don't need
\title{Curriculum Vitae}
\address{202,plot 78, Sec-19/c}{Koparkhairne, Navi mumbai}
\mobile{(+91) 9819316032}
\email{arvind.m2@tcs.com}
\homepage{arvibd.me}{\color{cyan}https://arvibd.me} % The first argument is the url for the clickable link, the second argument is the url displayed in the template - this allows special characters to be displayed such as the tilde in this example
\photo[80pt][0.4pt]{pictures/dsc.jpg} % The first bracket is the picture height, the second is the thickness of the frame around the picture (0pt for no frame)
\quote{"Let's live our dreams alive"}

%----------------------------------------------------------------------------------------

\begin{document}

\makecvtitle % Print the CV title

%--------------------------------------------------------------------

\section{Major Project}

\cvitem{Title}{\emph{Smart conversational shopping assistant}}
\cvitem{Description}{A conversational shopping assistant AI bot that helps user buy product at the best price from various providers. It remembers the user's preference and helps by giving recommendation. The bot uses NLP and Machine Learning for the conversational flow, Query optimisation and scaline decomposition for recommendation on incomplete dataset. The backend is based on django and processing is done via \emph{scikit-learn} and \emph{tensor Flow}. The dialog flow is based on \emph{Google dialog flow}. Springer paper link for the same will be available after peer review. The patent for the same is pending.}

%----------------------------------------------------------------------------------------
%	Projects
%----------------------------------------------------------------------------------------


\section{Paper}


\cventry{title}{Alfred - Conversational Shopping AI bot}{\textsc{NLP, Query optimisation, nltk}}{}{}{ A Springer paper published on conversational shopping assistant bot. \emph{Springer paper} is under peer review. First draft of the 
\href{https://www.dropbox.com/s/xkgjpawjezbnm6g/paper.pdf?dl=0
}{\color{blue} IEEE} paper }
\cventry{title}{2FA authentication mechnaism}{\textsc{MFA, 2FA, hashing, timestamp}}{}{}{ A paper on python cryptographic hash function and use of multi-factor authentication in various standard libraries}

\cventry{title}{SMBCRY exploit}{\textsc{wannacry, SMB protocol, bug}}{}{}{WannaCry affected almost millions of computers. Using a simple exploit in the SBM protocol in CIFS (windows File server). A detailed study and simulating the whole process in a sandbox environment.}


\section{Projects and Internships}


\cventry{2018}{ABM \& Brothers}{\textsc{droplets, DNS, digital Ocean, Cloud Computing}}{}{}{ Deployed a invoicePlan site for a Transport company named \href{http://139.59.85.218/ip/index.php/}{\color{blue}ABM \& Brothers} on digital Oceans.}

\cventry{project}{Social Network}{\textsc{Flask Social Network}}{}{}{A social platform for giving feedback and texting other users }

\cventry{iOS}{Fed-Or-Not-FedHead}{\textsc{iOS UI Basic}}{}{}{
iOS app for a Federer fans test. Just a basic quiz app. \href{https://github.com/Arvind2222/fedOrNotFed}{\color{blue}github repo} }

\cventry{django}{Learning Log}{\textsc{python Django}}{}{}{
The app for the academics. Very useful for thesis and those who want to summarise their studies everyday. Automatic feeds from the amazon goodreads \href{https://github.com/Arvind2222/learning-Log}{\color{blue}Learning-log} }

\cventry{django}{Django Blog}{\textsc{python Django}}{}{}{Blog using Django with MVC framework. My First website developed using a MVC framework. Learned how to configure and host on a heroku. }

\cventry{iOS}{breakpoint}{\textsc{iOS Firebase }}{}{}{Simple iOS game with firebase as backend \href{https://github.com/Arvind2222/breakpoint}{\color{blue}Breakpoint app repo} }


\cventry{project}{A do list app}{\textsc{Ruby on rails}}{}{}{A to-do list app on ROR with very basic functionality for CRUD. Developed as a part of learning ROR. }

%\cventry{project}{Scheduler iOS app}{\textsc{iOS, management }}{}{}{iOS app for managing your personal documents and notes. Makes tags to store the document securely and documents pdfs for reading on the go with scheduler }



\cventry{2011--2016}{legfeet.com}{\textsc{DNS, domain registrations, A-records}}{}{}{ I \emph{had} a website - legfeet.com (- every thing at your feet), The Site was used as an ecommerce platform using woo-commerce. And have been maintaining it since then. Based on the Wordpress installation. The site was active for 8 years. I created the site in my 9$^th$ grade and was in production till 2016}

\cventry{project}{Blog reader app}{\textsc{Android app, JSON, Wordpress plugins}}{}{}{ I have a website - \href{lyrianablog.wordpress.com}{\color{blue}lyrianablog} The website contains poems and writings with good tone towards life as an adult. I needed an app for the same so users can view the same. A blog reader app reading the JSON from the site and rendering it same to the view was the way to go}

\cventry{Social welfare}{offline wikipedia}{\textsc{SQL Dumps, CMS, Wikimedia}}{}{}{ Wikipedia releases the content every day in form of tar files. They call it dumps. There's a method to make the site available offline. I used of the methods to provide offline Wikipedia in rural area for children. This was mentioned and verified by Teach for India. And even appreciated and was given internship for the same}

\cventry{project}{Gprofiler}{\textsc{Github API, Angular, HTML}}{}{}{ 
	Github Profile search with basic Angular using Github API}

\cventry{\texttt{edx}}{Linux}{\textsc{SSH, samba, Shell, filesystems}}{}{}{ Passed with 90\% \href{https://verify.edx.org/cert/4849aee4d06b4a7a9bc7f64a39e597d7}{\color{blue} certificate verify}}

%--------------------------------------------------------------------

\section{Programming Languages}

\cvitem{Langauge}{\textsc{Python, C++, Java}}
\cvitem{ Markup }{\textsc{{\LaTeX}, HTML,CSS, Markdown}}
\cvitem{VersionControl}{\textsc{GIT}}
\cvitem{Web }{\textsc{HTML, CSS}}
\cvitem{OS}{\textsc{ Windows, Linux, macOS, open Solaris}}


\section{Programming skills}
\cvitem{OS}{I have a hands-on experience in operations on various Open-source projects and OS like Sun solaris, I have handled Debian and other Linux variants. I used macOS as a main OS and ubuntu with i3 dwm.}

\cvitem{algorithms}{I am always aware of the performance runtimes and their impact on the systems. My \href{https://www.codechef.com/users/arvind2222}{\color{blue}Codechef} profile}

%\section{Certification}

\section{Books and drafts}
\cvitem{Poem}{ Some of the poems on self and topic that are close to my heart. The book is available at \href{https://www.amazon.com/Youth-Poems-rooted-Arvind-Mishra-ebook/dp/B07CX46F4H/ref=sr_1_1?ie=UTF8&qid=1529053424&sr=8-1}{\color{blue} amazon} }{}
\cvitem{Non Fiction}{Currently in progress. About the socio and economic impact on the livings of the person without a conscience}{}

\section{Presentations}

\cvitem{AlphaZero beamer }{Deep mind alpha zero and the the brilliant use of the reinforcement Learning Algorithm  \href{https://www.dropbox.com/s/839fmc9ck3oqyxh/main.pdf?dl=0}{\color{blue}AlphaZero} presentation. Please have a look.A very informative and beautifully typeset in \LaTeX}

\cvitem{SMB beamer }{Deep mind alpha zero and the the brilliant use of the reinforcement Learning Algorithm  \href{https://www.dropbox.com/s/839fmc9ck3oqyxh/main.pdf?dl=0}{\color{blue}AlphaZero} presentation. Please have a look.A very informative and beautifully typeset in \LaTeX}

\section{Achievements and Award}
\cvitem{Jan -- 2018}{Top 1500 Rank in world - Codechef Jan Long Challenge - \href{https://www.codechef.com/users/arvind2222}{\color{blue} Jan Long}}

\cvitem{Oct -- 2016}{Bronze Medal in Hackerrank's Week of Code 24 }
\cvitem{Jun -- 2017}{Secured 2591 Rank in Snackdown 2017 Codechef \href{https://www.dropbox.com/s/za3d0gbiuwqe9ip/snackdown.pdf?dl=0}{\color{blue}Certificate}}

\cvitem{Innovation}{Google Science Fair participation certificate for designing better alternative to electrical system and transportation system \href{https://www.dropbox.com/s/31mhtjnwxqmn5r9/google.pdf?dl=0}{\color{blue}Certificate}}

\cvitem{Apr 06, 2017}{Certificate of Excellence in PHP -Techgig }{}

\cvitem{Running}{From 125Kg in my second year to a fit 15KM 78 Kg Runner. It taught me importance of \emph{discipline} in life than anything else. 
	I am BLUE \texttt{(ran more than 1000KM)} runner on NRC. I complete 100KM run a month.}

%----------------------------------------------------------------------------------------
%	LANGUAGES SECTION
%----------------------------------------------------------------------------------------
%
%\section{Languages}
%
%\cvitemwithcomment{English}{Fluent}{}
%\cvitemwithcomment{Hindl}{Fluent}{}

%----------------------------------------------------------------------------------------
%	INTERESTS SECTION
%----------------------------------------------------------------------------------------

\section{Interest}

\cvitem{Competitive Programming}{I am very much interested in solving problems. I compete on hackerrank and \href{https://www.codechef.com/users/arvind2222}{\color{blue} codechef} and codeforces. I use git for my projects and all the projects mentioned are on the \href{https://github.com/Arvind2222}{\color{blue}Github} as a public repo. Excluding some, they are ongoing projects and are as a private repository of github.   \href{https://bitbucket.org/arvibd}{\color{blue}bitbucket} }
\cvitem{Chess}{I like playing chess very much. Please visit my \href{https://www.dropbox.com/s/sv8rbnymswde53m/chess.png?dl=0}{\color{blue}chess24} profile }

\cvitem{Composition}{I like composition and it's design impact.\href{https://legfeet.wordpress.com}{\color{blue} Photography} }


\fancyfoot[l]{\parbox[b]{.8\textwidth}{\color{color2}\addressfont\itshape Proudly made in 
\href{https://www.dropbox.com/s/2zphv58tsk08q04/code.pdf?dl=0}{\color{blue}\LaTeX}}}


\end{document}



